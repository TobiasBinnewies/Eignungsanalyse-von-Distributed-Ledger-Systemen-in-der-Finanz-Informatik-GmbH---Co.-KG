
\noindent %um die Einrückung zu löschen

\section{Einleitung}

In dieser Arbeit wird auf die Eignung bzw. auf Use Cases von Distributed Ledger Systemen in der Finanz Informatik eingegangen.
Speziell wird dabei der Zahlungsverkehr betrachtet, dieser ist das Kerngeschäft der Sparkassen und so ein wichtiger Teil der FI.
Aktuell wird für diesen ein Großrechner verwendet.
Allerdings gibt es schon seit vielen Jahren das Ziel diesen abzuschaffen und durch eine andere Technologie zu ersetzten.
Daher könnte ein DLS eine Alternative darstellen.

\noindent
Die Arbeit befindet sich im Scope der FI, es wird also nicht der gesamte Zahlungsverkehr betrachtet, sondern nur die konkrete Implementierung des Zahlungsverkehrs in der FI.

\noindent
Zunächst werden DLS im allgemeinen betrachtet und erklärt, um ein Grundverständnis für diese Technologie zu schaffen.
Dann folgt eine Betrachtung des Ist-Zusands des Zahlungsverkehrs, um die Anforderungen an diesen zu ermitteln.
Anschließend werden Möglichkeiten der Verwendung von DLS im Zahlungsverkehr betrachtet.
Abschließend folgt eine Schlussfolgerung und ein Ausblick auf weitere mögliche Verwendungsmöglichkeiten von DLS.
\noindent
\section{Praxis}

% \noindent

\subsection{Bewertungskriterien}

\todoin[color=NOTES]{
Kriterien: \break
- Perfomance (Schnelligkeit) --> Durch Layer 2 Lösungen (z.B. Lightning Network) (recht) hoch, ansonsten durch privates System (dann aber nicht dezentral und so Sicherheitsrisiken) \break
- Skalierbarkeit --> Durch Layer 2 Lösungen (z.B. Lightning Network) möglich \break
- Sicherheit (u.a. AuditLog) --> Wenn öffentliches System, dann generell durch Implementeirung beeinflusst, ansonsten durch privates System wie jetzt \break
- Transparenz --> In öffentlichem System sehr hoch, ansonsten durch privates Systeme wie jetzt \break
- Privatsphäre --> In öffnetlichem System quasi nicht möglich (evtl. durch HD-Wallets), ansonsten durch privates System (dann aber nicht dezentral und so Sicherheitsrisiken) \break
- Kosten  \break
- Komplexität (Aufwand, Wartung, ...) --> Erstmal Aufwand, dann aber gringer (vor allem wenn nur ein System verwendet wird, im Gegensatz zur aktuellen Situation) \break
- Erweiterbarkeit (z.B. neue Anforderungen) --> Durch Proxys möglich \break
}

\todoin[color=NOTES]{
    Ist-Zustand: \break
Quelle: Persönliches Gespräch mit Nils Pr... (OE-XXXX) \break
Portal (Sparkassenmitarbeiter) \break
Online-Banking (Sparkassenkunde) \break
Cobol-Job \break
$\downarrow$ \break
OSPE-Services \break
$\downarrow$ \break
DOING 1 - N \break
$\downarrow$ \break
Inhausclearing: DiBus (Quelle: Gespräch Christian Krauthoff (OE-XXXX)) \break
Inland / SEPA: Clearing / TARGET / SWIFT \break
Ausland: Clearing / SWIFT \break
\break
"Eigentliche Idee von Cobol wegkommen --> Neues Projekt (AZV) nun aber 80\% Cobol" \break
\break
\break
\break
Fragen:\break
- Wie wird sichergestellt das nur der Inhaber eines Kontos Buchungen durchführen kann? \break
- Wie wird sichergestellt, dass die Datenbank mit den Kontodaten nicht manipuliert werden kann? \break
- Kosten für Transaktionen / Kontoführung generell? \break
Osplus produktkatalog \break

- Wie wird der Zugang zur Prod-DB geregelt? \break
\break
\break
\break
Sicherheitsaspekte: \break
- Code Review durch unabhängige Entwickler (also welche die nicht am Projekt mitgearbeitet haben), Quelle: Gespräch mit Paulina ... (OE-XXXX) \break
- Automatische Codeprüfung, Quelle: Gespräch mit Paulina ... (OE-XXXX) \break
- GPS (Geschäftsprozesse), die die Verwendung von Daten in einem bestimmten Kontext klar definieren, Quelle: Gespräch mit Paulina ... (OE-XXXX) \break
}

\todoin[color=NOTES]{
Quelle: Wiki - Richtlinie Geschäftsprozesse \break
Unterhalb dieser Seite finden Sie eine Artikel zum Thema Geschäftsprozesse. Unter "Geschäftsprozess" sind hier die mit der Anwendung GPA erstellten Abläufe durch eine Verkettung von Aktivitätstypen zu verstehen. Diese Geschäftsprozesse werden durch die GPS ausgeführt, um ein bestimmtes geschäftliches oder betriebliches Ziel zu erreichen.
Die Beschreibungen der untergeordneten Seiten haben jeweils einen unterschiedlichen Charakter und dienen der Information oder stellen Richtlinien für die Entwicklung, den Test und die Ausführung von Geschäftsprozessen dar. Details finden Sie jeweils auf den einzelnen Seiten.
Die Seiten werden fortlaufend im Rahmen eines OSPlus-Releases und der Weiterentwicklung der Anwendungen GPA und GPS (siehe Glossar) aktualisiert.
Untergeordnete Seiten:
}
\noindent

\section{Ausblick}
\label{sec:ausblick}

Die grundlegende Idee eines DLS ist es, dessen Teilnehmer zu vernetzten, ohne zentrale Instanzen zu verwenden und so kein Vertrauen in diese setzen zu müssen (s. \ref{sec:definition-kryptowaehrung}).
Da in dieser Arbeit die Verwendung von DLS lediglich im Scope einer einzelnen Instanz - der FI - betrachtet wurde, und daher keine Vernetzung stattfindet, fällt dieser Vorteil vollständig weg.
Eine weitaus sinnvollere Verwendung im Berech Zahlungsverkehr wäre die Vernetzung mehrerer Banken, um so bpsw. Transaktionen auszuführen.
Dafür kann so auch ein privates Netzwerk geschaffen werden, da dies dann auch dezentral wäre und so die Vorteile eines DLS genutzt werden können.
Die Bundesbank hat hierzu einen Bericht geschrieben, in dem sie sich positiv für diese Technologie im Auslandszahlungsvekehr äußert, um diesen so zu vereinfachen und zu beschleunigen. \footcite[Vgl.][44]{q12}
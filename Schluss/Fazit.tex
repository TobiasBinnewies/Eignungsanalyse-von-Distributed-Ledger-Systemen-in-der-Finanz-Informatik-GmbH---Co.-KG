\section{Schlussfolgerung}
\label{sec:Fazit}

Die Verwendung von DLS also reine Datenbank verfehlt den Zweck, den Großrechner abzulösen. 
Hinzu kommt, dass die Verwendung als Datenbank auch keine sonderlich gute Alternative ist, da dies mit einem sehr hohem Aufwand und Komplexität verbunden ist und die Verwendung teurer und langsamer ist.
Außerdem besteht auch kein Problem mit der aktuellen Art und Weise wie Daten gespeichert werden, dass mit DLS gelöst werden würde.

\noindent
Die Verwendung von Optimistic Rollups als Ersatz für den Großrechner ist schon eher eine Alternative, allerdings auch sehr komplex.
Die Verwendung von DLS in Verbindung mit Cloud Computing könnte allerdings eines der größten Probleme, die mit Cloud Computing bestehen - die Datensicherheit -, lösen.
Allerdings ist es fraglich, ob diese Technologie die Performance des Großrechners erreichen kann.
Dafür müsste diese Idee prototypisch umgesetzt und getestet werden, was aufgrund der hohen Komplexität und des benötigten Testdatenumfangs mit einem sehr hohen Aufwand verbunden wäre.
%Es bestehen hohe Zweifel, ob andere Technologien den Großrechner ersetzen können.\footappendix{Vgl.}{i1:f4}

\noindent
Daher ist die Verwendung von DLS für den Zahlungsverkehr der Sparkassen eher wenig geeignet.
Wenn, dann nur in Verwendung mit Cloud Systemen.
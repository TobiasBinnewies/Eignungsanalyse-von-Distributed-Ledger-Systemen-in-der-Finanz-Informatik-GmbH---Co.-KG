\noindent

\subsection{Distributed Ledger System}
\label{sec:definition-dls}

\subsubsection{Distributed Ledger}
\label{sec:definition-distributed-ledger}
Ein Distributed Ledger ist eine Datenbank, die im Konsens geteilt und über ein Netzwerk synchronisiert wird, das sich über mehrere Standorte, Institutionen oder Länder erstreckt.\footcite[Vgl. hierzu und im Folgenden][]{w1,w2} 
Es ermöglicht, dass Transaktionen und Aufzeichnungen öffentlich und überprüfbar sind, und da es dezentralisiert ist, gibt es keinen einzelnen Ausfallpunkt. 
Jeder Teilnehmer im Netzwerk hat Zugang zu den Aufzeichnungen, die über dieses Netzwerk geteilt werden, und kann eine identische Kopie der Daten haben. Änderungen oder Ergänzungen am DL werden nahezu in Echtzeit in allen Kopien widergespiegelt, was die Transparenz und Sicherheit erhöht.
\bigbreak
\noindent
Ein Distributed Ledger System ist ein System, das einen Distributed Ledger verwendet, um Transaktionen zwischen Teilnehmern zu verwalten. Am häufigsten wird die Blockchain-Technologie als DL verwendet.\footcite[Vgl.][]{w3}
% \bigbreak
% \noindent
\subsubsection{Blockchain}
\label{sec:definition-blockchain}
% REVIEW evtl. ausführlicher
Eine Blockchain ist eine spezielle Form eines Distributed Ledgers, die aus einer Kette von Blöcken besteht, die jeweils einen Hash des vorherigen Blocks enthalten.\footcite[Vgl. hierzu und im Folgenden][16]{q3} 
So wird gewährleistet, dass die Blöcke - und so auch die darin beinhalteten Daten - nach ihrem Eintrag nicht verändert werden können, ohne alle nachfolgenden Blöcke abzuändern.

\subsubsection{Kryptowährung}
\label{sec:definition-kryptowaehrung}
Ein DLS kann viele Anwendungsmöglichkeiten haben. Am wohl bekanntesten ist die Verwendung als Kryptowährung, wie bspw. Bitcoin.\footcite[Vgl. hierzu und im Folgenden][1]{q4} 
Hierbei werden die Transaktionen zwischen den Teilnehmern des Netzwerks durchgeführt, die in einer Blockchain festgehalten werden. 
So können Transaktionen Peer-to-Peer durchgeführt werden, also ohne dass eine zentrale Instanz diese überprüfen muss, da die Korrektheit der Transaktionen von allen Teilnehmern geprüft werden.
\bigbreak
\noindent
Bestimmste Chains - wie EVM-Chains - heben dieses Konzept auf eine neue Ebene, indem sie es ermöglichen, Smart Contracts zu erstellen.

\subsubsection{Ethereum Virtual Maschine Chain}
\label{sec:definition-evm-chain}
Eine EVM-Chain ist eine Blockchain-Netzwerk, das die Ethereum Virtual Machine (EVM) verwendet, um Smart Contracts auszuführen.\footcite[Vgl. hierzu und im Folgenden][]{w5}
Ethereum selbst ist eine EVM-Chain, die die Kryptowährung Ether (ETH) verwendet. Allerdings gibt es noch weitere Chains, die ebenfalls kompatibel mit der EVM sind, wie bspw. Binance Smart Chain (BSC) oder Polygon (MATIC)\footcite[Vgl.][]{w6}.
So kann für jede dieser Chains der gleiche Code - sowie Tools für dessen Entwicklung - verwendet werden, um Smart Contracts zu erstellen, die dann auf der jeweiligen Chain ausgeführt werden.

\subsubsection{Smart Contracts}
\label{sec:definition-smart-contracts}

Smart Contracts (im Sinne von Ethereum) sind Programme, die auf der Ethereum-Blockchain ausgeführt werden.\footcite[Vgl. hierzu und im Folgenden][]{w4} 
Sie bestehen aus einer Sammlung von Code (ihre Funktionen) und Daten (ihr Zustand), die - ebenso wie ein Benutzer-Wallet - an einer bestimmten Adresse auf der Ethereum-Blockchain (oder einer anderen EVM-Chain - S. \ref{sec:definition-evm-chain}) residieren.
Smart Contracts sind eine Art von Ethereum-Konto, was bedeutet, dass sie ein Guthaben haben und Ziel von Transaktionen sein können. 
Sie werden jedoch nicht von einem Benutzer kontrolliert, sondern sind im Netzwerk bereitgestellt und laufen wie programmiert. 
Benutzerkonten können dann mit einem Smart Contract interagieren, indem sie Transaktionen einreichen, die eine auf dem Smart Contract definierte Funktion ausführen. 
Außerdem können Regeln / Bedingungen definiert werden, nach den Code automatisch durchgeführt wird.
Standardmäßig können Smart Contracts nicht gelöscht werden, und Interaktionen mit ihnen sind irreversibel.

% REVIEW Gas erklären?

\subsubsection{ERC20 Token}
\label{sec:definition-erc20-token}
Es gibt diverse Standarts für Smart Contracts, die bestimmte Funktionen und Eigenschaften definieren.
Einer dieser Standarts ist der ERC20 Token Standard, der die Schnittstellen eines Smart Contracts definiert, der als Token verwendet werden soll.\footcite[Vgl. hierzu und im Folgenden][]{w7}
Ein Token kann dabei eine beliebige Repräsentation eines Vermögenswertes sein.
In diesem Smart Contract wird die Anzahl der Token gespeichert, die eine bestimme Adresse (also Benutzer-Wallet oder Smart Contract) besitzt.
Außerdem werden Funktionen definiert, um u. a. Token von einer Adresse zu einer anderen zu versenden, die Anzahl der Token einer Adresse abzufragen und anderen Adressen die Erlaubnis zu erteilen, Token von der eigenen Adresse zu versenden.\footcites[Vgl.][]{w8}[]{w7}

% NOTES Proxys?
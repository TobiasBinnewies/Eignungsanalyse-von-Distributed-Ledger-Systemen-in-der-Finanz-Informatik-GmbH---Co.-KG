\subsection{Blockchain}
\label{sec:definition-blockchain}
\todo[color=REVIEW]{evtl. ausführlicher}
Eine Blockchain ist eine spezielle Form eines Distributed Ledgers, die aus einer Kette von Blöcken besteht, die jeweils die zu speichernden Daten enthalten.\footcite[Vgl.][16]{q3} 
Ein Block besteht aus einem Header und einem Body.\footcites[Vgl. hierzu und im Folgenden][S. 161 ff\adddot]{q5}[]{w9}
\bigbreak
\noindent
Der Header enthält u. a. den Hash des vorherigen Blocks, einen Zeitstempel und die Nummer des Blocks.
Außerdem enthält der Header einen Hash des Bodys.
So wird gewährleistet, dass die Blöcke - und so auch die darin beinhalteten Daten - nach ihrem Eintrag nicht verändert werden können, ohne alle nachfolgenden Blöcke abzuändern. 
(So wird von „Block-Confirmation” gesprochen, wenn eine bestimmte Anzahl von Blöcken nach diesem Block hinzugefügt wurden - je mehr desto sicherer -, da er erst dann als „unveränderlich” gilt.\footcites[Vgl.][191]{q5})
\bigbreak
\noindent
Der Body enthält die eigentlichen Daten, die gespeichert werden sollen. Im Falle einer Kryptowährung sind dies bspw. die Transaktionen, die in diesem Block gespeichert werden.
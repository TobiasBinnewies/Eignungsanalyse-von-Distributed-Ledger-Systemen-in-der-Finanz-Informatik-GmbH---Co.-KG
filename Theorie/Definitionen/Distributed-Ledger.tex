\subsubsection{Distributed Ledger}
\label{sec:definition-distributed-ledger}
Ein Distributed Ledger ist eine Datenbank, die im Konsens geteilt und über ein Netzwerk synchronisiert wird, das sich über mehrere Standorte, Institutionen oder Länder erstreckt.\footcite[Vgl. hierzu und im Folgenden][]{w1,w2} 
Es ermöglicht, dass Transaktionen und Aufzeichnungen öffentlich und überprüfbar sind, und da es dezentralisiert ist, gibt es keinen einzelnen Ausfallpunkt. 
Jeder Teilnehmer im Netzwerk hat Zugang zu den Aufzeichnungen, die über dieses Netzwerk geteilt werden, und kann eine identische Kopie der Daten haben. Änderungen oder Ergänzungen am DL werden nahezu in Echtzeit in allen Kopien widergespiegelt, was die Transparenz und Sicherheit erhöht.
\bigbreak
\noindent
Ein Distributed Ledger System ist ein System, das einen Distributed Ledger verwendet, um Transaktionen zwischen Teilnehmern zu verwalten. Am häufigsten wird die Blockchain-Technologie als DL verwendet.\footcite[Vgl.][]{w3}
\subsection{Ethereum Virtual Maschine Chain}
\label{sec:definition-evm-chain}
EVM-Chains heben das Konzept DLS auf eine neue Ebene, indem sie es ermöglichen, nicht nur Transaktionen abzuwicklern, sondern dazu noch vorprogramierten Code (Smart Contracts) auszuführen und so viele neue Anwendungsmöglichkeiten erschaffen.\footcite[Vgl. hierzu und zum Folgenden][]{w5}
Eine EVM-Chain ist also eine Blockchain-Netzwerk, das die Ethereum Virtual Machine (EVM) verwendet, um Smart Contracts auszuführen.
Ethereum selbst ist eine EVM-Chain, die die Kryptowährung Ether (ETH) verwendet. Allerdings gibt es noch weitere Chains, die ebenfalls kompatibel mit der EVM sind, wie bspw. Binance Smart Chain (BSC) oder Polygon (MATIC)\footcite[Vgl.][]{w6}.
\todo[color=REVIEW]{Polygon \& Binance hier streichen (ist Layer 2)}
So kann für jede dieser Chains der gleiche Code - sowie Tools für dessen Entwicklung - verwendet werden, um Smart Contracts zu erstellen, die dann auf der jeweiligen Chain ausgeführt werden können.
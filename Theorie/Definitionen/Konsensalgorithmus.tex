\subsection{Konsensalgorithmus}
\label{sec:definition-konsensalgorithmus}
Um sicherzustellen, dass bestehende Blöcke nicht verändert werden können und der Inhalt neuer Blöcke valide ist, wird ein Konsensalgorithmus verwendet.\footcite[Vgl.][S. 2 f\adddot]{q4}
Die bekanntesten Algorithmen sind:
\begin{itemize}
    \item \textbf{Proof of Work (PoW):}
    Miner (Nodes) konkurrieren miteinerander, um ein mathematisches Problem zu lösen.\footcite[Vgl. hierzu und im Folgenden][3]{q4}
    Dem ersten, dem dies gelingt, erhält die Blockbelohnung und kann den nächsten Block erstellen.
    Um einen Block zu erstellen wird also Zeit benötigt, ein Angreifer müsste also eine höheren Rechenleistung als die Hälfte des Netzwerks besitzen, um die Blockchain zu manipulieren.

    \item \textbf{Proof of Stake (PoS):}
    Es wird zufällig eine Node ausgewählt, die den nächsten Block erstellen darf und so die Belohnung erhält.\footcite[Vgl. hierzu und im Folgenden][2]{q7}
    Es muss vorab eine Sicherheitsleistung (Stake) hinterlegt werden, die verloren geht, wenn die Node versucht, die Blockchain zu manipulieren. 
    Je höher der Stake, desto höher die Wahrscheinlichkeit, dass die Node ausgewählt wird.
    So wird verhindert, dass ein Angreifer die Blockchain manipuliert, da er mehr als die Hälfte des Stakes besitzen müsste, um die Blockchain zu manipulieren.

    \item \textbf{Proof of Authority (PoA):}
    Es wird eine Liste von Nodes festgelegt, die die Blockchain validieren dürfen.\footcite[Vgl. hierzu und im Folgenden][]{w14}
    Dieser Algorithmus wird häufig bei privaten Chains verwendet, da den Nodes vertraut werden muss.
    So kann die Blockchain nicht manipuliert werden, ohne dass eine der Nodes dies zulässt.
    Dieser Algorithmus ist sehr schnell, da keine Rechenleistung benötigt wird oder eine Auswahl getroffen werden muss, um einen Block zu erstellen.
    
\end{itemize}
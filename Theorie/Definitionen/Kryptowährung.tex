\subsubsection{Kryptowährung}
\label{sec:definition-kryptowaehrung}
Ein DLS kann viele Anwendungsmöglichkeiten haben. Am wohl bekanntesten ist die Verwendung als Kryptowährung, wie bspw. Bitcoin.\footcite[Vgl. hierzu und im Folgenden][1]{q4} 
Hierbei werden die Transaktionen zwischen den Teilnehmern des Netzwerks durchgeführt, die in einer Blockchain festgehalten werden. 
So können Transaktionen Peer-to-Peer durchgeführt werden, also ohne dass eine zentrale Instanz diese überprüfen muss, da die Korrektheit der Transaktionen von allen Teilnehmern geprüft werden.
\bigbreak
\noindent
Bestimmste Chains - wie EVM-Chains - heben dieses Konzept auf eine neue Ebene, indem sie es ermöglichen, Smart Contracts zu erstellen.
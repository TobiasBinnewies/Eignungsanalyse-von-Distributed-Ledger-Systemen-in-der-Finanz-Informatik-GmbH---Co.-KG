\subsection{Smart Contracts}
\label{sec:definition-smart-contracts}
Smart Contracts (im Sinne von Ethereum) sind Programme, die auf der Ethereum-Blockchain ausgeführt werden.\footcite[Vgl. hierzu und zum Folgenden][]{w4} 
Sie bestehen aus einer Sammlung von Code (ihre Funktionen) und Daten (ihr Zustand), die - ebenso wie ein Benutzer-Wallet - an einer bestimmten Adresse auf der Ethereum-Blockchain (oder einer anderen EVM-Chain - S. \ref{sec:definition-evm-chain}) residieren.
Smart Contracts sind eine Art von Ethereum-Konto, was bedeutet, dass sie ein Guthaben haben und Ziel von Transaktionen sein können. 
Sie werden jedoch nicht von einem Benutzer kontrolliert, sondern sind im Netzwerk bereitgestellt und laufen wie programmiert. 
Benutzerkonten können dann mit einem Smart Contract interagieren, indem sie Transaktionen einreichen, die eine auf dem Smart Contract definierte Funktion ausführen. 
Außerdem können Regeln / Bedingungen definiert werden, nach denen Code automatisch durchgeführt wird.
Standardmäßig können Smart Contracts nicht gelöscht werden, und Interaktionen mit ihnen sind irreversibel.
\subsection{Anonymität}
In einem öffeltichen Blockchain-Netzwerk sind alle Transaktionen öffentlich einsehbar.
Anonymität wird dadurch gewährleistet, dass die Identität eines Teilnehmers von der Walletaddress getrennt ist.\footcite[Vgl.][6]{q4}
Allerdings können mehrere Transaktionen eines Wallets miteinander in Verbindung gebracht werden, und so die Anonymität eines Teilnehmers gefährden. So wird - zumindest bei Bitcoin - dazu geraten jede Address nur genau zweimal zu verwenden. Einmal um Bitcoin zu empfangen und einmal um dieses wieder zu versenden.\footcite[Vgl.][]{w15}
Dieses Problem würde ebenfalls bestehen, wenn die Kontostände in einem öffentlichen Netzwerk abgebildet werden würden.

% TODO Unterscheid zwischen Account- und Transactions-basierten Chains erklären
% --> Account-based Problem: Keine HD-Wallets einfach möglich, so weniger Anonymität
\subsubsection{Account-based vs. UTXO-based Chains}
\label{sec:definition-account-based-vs-utxo-based-chains}

\subsubsection{HD-Wallets}
Bei einem HD-Wallet werden beliebig viele Keys (sprich Adressen) aus einem Hauptschlüssel (Seed) generiert.\footcite[Vgl. hierzu und weiterführend][S. 8 ff\adddot]{q9}
So kann ein Wallet verwendet werden, dabei aber für jede Transaktion eine neue Adresse verwendet werden und so eine hohe Anonymität gewährleistet werden.
Diese Art von Wallet sind der heutige Standart für UTXO-based Chains.
Allerdings können HD-Wallets nicht für Account-based Chains verwendet werden, da diese keine Keys verwenden, sondern die Anzahl an Coins pro Account speichern.

% TODO Hier Idee mit UTXO-based Token auf Account-based Chain einfügen (bzgw. HD-Wallet fähigen Token)
% --> UTXO vermutlich besser, weil weniger Chain Einträge / Auslesungen notwendig sind
\subsubsection{UTXO-based Token}



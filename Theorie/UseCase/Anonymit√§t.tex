\subsection{Anonymität}
In einem öffeltichen Blockchain-Netzwerk sind alle Transaktionen öffentlich einsehbar.
Anonymität wird dadurch gewährleistet, dass die Identität eines Teilnehmers von der Walletaddress getrennt ist.\footcite[Vgl.][6]{q4}
Allerdings können mehrere Transaktionen eines Wallets miteinander in Verbindung gebracht werden, und so die Anonymität eines Teilnehmers gefährden. So wird - zumindest bei Bitcoin - dazu geraten jede Address nur genau zweimal zu verwenden. Einmal um Bitcoin zu empfangen und einmal um dieses wieder zu versenden.\footcite[Vgl.][]{w15}
Dieses Problem würde ebenfalls bestehen, wenn die Kontostände in einem öffentlichen Netzwerk abgebildet werden würden.

\paragraph{Account-based vs. UTXO-based Chain}
\label{sec:definition-account-based-vs-utxo-based-chain}
\noindent
Bei einer UTXO-based Chain (bspw. Bitcoin) werden die Anzahl an Coins pro offener Transaktion gespeichert.\footcite[Vgl. hierzu und zum Folgenden sowie weiterführend][S. 182 ff\adddot]{q5}
Eine Adresse besitzt also genauso viele Coins wie die Summe der offenen Transaktionen, die an diese Adresse gesendet wurden.
Eine Transaktion kann mehrere Ein- und Ausgänge haben, wobei die Summe gleich sein muss.
So können auch zwei offene Transaktionen unterschiedlicher Adressen zusammengefasst und einem Empfänger zugesendet werden.
\bigbreak
\noindent
Bei einer Account-based Chain (bspw. EVM-Chains) werden die Anzahl an Coins pro Account / Adresse gespeichert.\footcite[Vgl. hierzu und zum Folgenden][]{w16} 
So wird bei einer Transaktion die Anzahl an Coins von einem Account auf einen anderen Account übertragen, sprich, die Anzahl an Coins wird von einem Account abgezogen und dem anderen Account hinzugefügt.
Die Art von Chain wird vor allem bei Smart Contract fokussierten Blockchains verwendet.
Um in diesem Modell Coins mehrerer Adressen zusammenzufassen, sind mehrere Transaktionen notwendig.


\subsubsection{HD-Wallets}
Bei einem HD-Wallet werden beliebig viele Keys (sprich Adressen) aus einem Hauptschlüssel (Seed) generiert.\footcite[Vgl. hierzu und weiterführend][S. 8 ff\adddot]{q9}
So kann ein Wallet verwendet werden, dabei aber für jede Transaktion eine neue Adresse verwendet werden und so eine hohe Anonymität gewährleistet werden.
Diese Art von Wallet sind der heutige Standart für UTXO-based Chains.
Allerdings können HD-Wallets nicht für Account-based Chains verwendet werden, da diese keine Keys verwenden, sondern die Anzahl an Coins pro Account speichern.

% TODO Hier Idee mit UTXO-based Token auf Account-based Chain einfügen (bzgw. HD-Wallet fähigen Token)
% --> UTXO vermutlich besser, weil weniger Chain Einträge / Auslesungen notwendig sind
\subsubsection{UTXO-based Token}
Aufbauend auf der Idee einen neuen Token zu kreieren (s. \ref{datenspeicherung:neuer-token}) ... % TODO Weiterschreiben

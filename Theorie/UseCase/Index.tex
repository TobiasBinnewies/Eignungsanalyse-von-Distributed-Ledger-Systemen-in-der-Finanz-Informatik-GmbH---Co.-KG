\noindent

\section{Use Case Zahlungsverkehr}

% \subsubsection{Zahlungsverkehr}

\todoin[color=NOTES]{
Zahlungsverkehr: \break
- Je nach Aufbau kann nur der Kontoinhaber selbst eine Transaktion vornehmen \break
- Transaktionen können nicht rückgängig gemacht werden --> Rückbuchung durch erneute Transaktion \break
- Transaktionen können nicht verändert werden --> Keine Manipulation möglich \break
- AuditLog ist automatisch durch Technologie vorhanden \break
- Durch Smart Contracts können bestimmte Finanzprodukte (z.B. Sparverträge, Kredite, ...) oder auch Multisign (per Multisign Contracts) abgebildet werden und so automatisiert werden \break
Großrechnerabschaffung \break
\break
Probleme: \break
- Transaktionen sind öffentlich einsehbar --> Keine Privatsphäre (außer evtl. durch HD-Wallets) \break
- Euro keine Kryptowährung --> Umwandlung in Token notwendig \break
- Weiterentwicklung bei bspw. neuen Anforderungen schwierig --> Durch Proxys möglich \break
}

\todo[color=REVIEW]{
    QUELLEN!?
}
Ein Use Case dieser Technologie in Bereich einer Bank wäre der Zahlungsverkehr. 
So wird die Blockchain als Datenbank für die Konten der Kunden verwendet.
Eine Blockchain erfüllt automatisch durch ihren Aufbau einige Anforderungen an den Zahlungsverkehr, die in herkömlichen Systemen beachtet und umgesetzt werden müssen. 
So können Transaktionen - also in diesem Fall Einträge in diese Datenbank - nicht rückgängig, nicht verändert und so nicht manipuliert werden. 
Es gäbe dennoch die Möglichkeit - je nach konkreter Implementierung - bestimmte Sicherheitsmechanismen einzubauen, um bspw. gegen Geldwäsche oder fehlerhaft Buchungen vorzugehen. \todo[color=TODO]{Ausführlicher \& Besipiele / konkrete Idee der Implementierung}
Außerdem ist ein AuditLog automatisch vorhanden, da alle Transaktionen in der Blockchain gespeichert werden.
Aufbauend darauf können Smart Contracts verwendet werden, um bestimmte Finanzprodukte (z.B. Sparverträge, Kredite, ...) oder auch Multisign (per Multisign Contracts) abzubilden und so zu automatisieren. \todo[color=TODO]{Ausführlicher \& Besipiele / konkrete Idee der Implementierung}
%...
\subsection{Datenspeicherung}
Um den Kontostand eines Kunden widerzuspiegeln, gibt es folgende Möglichkeiten.

\subsubsection{Layer 1 Währung}
Bei der Darstellung des Kontostandes als Layer 1 Währung (z. B. ETH) müsste jeder Kunde ein Wallet erhalten / besitzen, dass den Wert des Kontos in einer Layer 1 Währung enthält.
Das Problem dabei ist, dass diese nicht den Euro darstellt. 
So müsste der Wert immer in eine andere Währung umgerechnet werden, was zu zusätzlichen Kosten führt. Außerdem ist der Wert einer Layer 1 Währung sehr volatil, was zu Problemen führen kann, wenn der Wert des Kontos nicht mit dem Wert der Layer 1 Währung übereinstimmt.
Die Darstellungsart kommt also nicht in Frage.

\subsubsection{ERC20 Token}
\label{sec:definition-erc20-token}
Es gibt diverse Standarts für Smart Contracts, die bestimmte Funktionen und Eigenschaften definieren.
Einer dieser Standarts ist der ERC20 Token Standard, der die Schnittstellen eines Smart Contracts definiert, der als Token verwendet werden soll.\footcite[Vgl. hierzu und zum Folgenden][]{w7}
Ein Token kann dabei eine beliebige Repräsentation eines Vermögenswertes sein.
In diesem Smart Contract wird die Anzahl der Token gespeichert, die eine bestimme Adresse (also Benutzer-Wallet oder Smart Contract) besitzt.
Außerdem werden Funktionen definiert, um u. a. Token von einer Adresse zu einer anderen zu versenden, die Anzahl der Token einer Adresse abzufragen und anderen Adressen die Erlaubnis zu erteilen, Token von der eigenen Adresse zu versenden.\footcites[Vgl.][]{w8}[]{w7}

\paragraph{Stablecoin}
Es gibt bestimmte ERC20 Token, die den Wert anderer Assets (unter anderem auch den Euro) abbilden. Diese werden Stablecoin genannt.\footcite[Vgl. hierzu und weiterführend][4]{q8}
Das Problem daran - sowie auch bei der Layer 1 Währung - ist, dass diese einen tatsächlichen Wert haben, und so die Bank dieses Geld nicht für eigene Geschäfte verwenden kann.

\paragraph{Neuer Token}
\label{datenspeicherung:neuer-token}
Es wäre also die Erstellung eines eigenen Tokens sinnvoll, um den Wert eines Kontos darzustellen, ohne dass dieser einen tatsächlichen Wert hat. So kann die Bank diesen Wert für eigene Geschäfte verwenden, ohne dass der Kunde dadurch einen Verlust erleidet. 
Außerdem kann so gewährleistet werden, dass nur Kunden der Bank diesen Token verwenden können, da die Bank die einzige ist, die diesen Token ausgibt.
Außerdem muss der erstellte Token nicht zwingend die genauen Schnittstellen eines ERC20 Tokens, sondern lediglich die Anforderungen der Bank erfüllen.

\input{Theorie/UseCase/Anonymität.tex}

\noindent

\subsection{Öffentliche vs. private Chain}
\label{sec:oeffentlich-vs-privates-netzwerk}

\todoin[color=NOTES]{
    Öffentliches vs. privates Netzwerk: \break
    PRivates: \break
    - Nur für bestimmte Teilnehmer (in dem Falle dann nur FI) \break
    - wirds wirklich nur von einer Authority verwendet können die Daten geändert werden \break
    - "eigene" Regeln (bspw. kein Gas) \break
    \break
    Öffentliches: \break
    - Anonymität schwierig zu gewährleisten \break
    - Transaktionskosten \break
    - Daten können nicht geändert werden \break
}

Bei der Verwendung einer Blockchain gibt es die Möglichkeit, einer öffentlichen Chain „beizutreten“ oder dafür eine private Chain zu betreiben.\footcite[Vgl.][]{w10}
Eine öffentliche Chain ist für alle Teilnehmer offen und kann von jedem verwendet werden.\footcite[Vgl. hierzu und zum Folgenden][]{w11}
Eine privatee Chain hingegen ist nur für ausgewählte Teilnehmer zugänglich und wird i.d.R. von einer oder mehreren Organisationen / Unternehmen betrieben.
Die Auswahl der Art der Chain kann an den Punkten Sicherheit / Unveränderlichkeit, Leistung, Kosten, Berechtigungen und Datenschutz / Anonymität erfolgen:
\begin{itemize}
    \item \textbf{Sicherheit / Unveränderlichkeit:} 
    \todo[color=REVIEW]{Folgendes eher in Definition oder späterer Analyse?}
    Die Sicherheit und Unveränderlichkeit einer Blockchain wird durch ihren Konsensalgorithmus bestimmt. 
    Eine öffentliche Chain wird durch die Interaktion von Tausenden unabhängigen Nodes gesichert, die von Einzelpersonen und Organisationen auf der ganzen Welt betrieben werden. 
    Private Chains haben typischerweise eine kleine Anzahl von Nodes, die von einer oder wenigen Organisationen kontrolliert werden. 
    Diese Nodes können streng kontrolliert werden, aber nur wenige müssen kompromittiert werden, um die Chain umzuschreiben oder betrügerische Transaktionen durchzuführen.

    \item \textbf{Leistung:} 
    Bei privaten Chains können Hochleistungsnodes mit besondererer Hardware sowie ein anderer Konsensalgorithmus verwendet werden, um einen höheren Tansaktionsdurchsatz auf der Basisschicht (Layer 1) erreichen.
    Bei einer öffentlichen Chain kann ein hoher Durchsatz mit Hilfe von Layer 2 Skalierungslösungen erreicht werden.

    \item  \textbf{Kosten:}
    Die Kosten für den Betrieb einer privaten Chain spiegeln sich hauptsächlich in der Arbeit wider, die Chain einzurichten und zu verwalten, und den Servern zu betreiben, auf denen sie läuft. 
    Während es keine Kosten gibt, um sich mit dem Ethereum Mainnet zu verbinden, muss die Gas Fee (s. \ref{sec:definition-gas-fee}) für jede Transaktion in Ether bezahlt werden.
    Abhilfe können Transaction Relayers (s. \ref{sec:definition-transaction-relayers}) sein, sodass ein Endkunde diese Gebühr nicht selbst tragen muss.

    Einige Analysen haben gezeigt, dass die Gesamtkosten für den Betrieb einer Anwendung auf einer öffentlichen Chain niedriger sein können als beim Betrieb einer privaten Chain.\footcite[Vgl.][14]{q6}
    \todo[color=REVIEW]{Die Quelle genauer erläutern?}

    \item \textbf{Berechtigungen:}
    Bei privaten Chains können nur autorisierte Teilnehmer Nodes einrichten und Transaktionen durchführen.
    Bei öffentlichen Chains kann jeder Node einrichten und Transaktionen durchführen.
    So kann ebenfalls jeder auf jeden Contract zugreifen, also dessen gespeicherte Daten auslesen und Funktionen aufrufen.
    Daher müssen erstellte Contracts so implementiert werden, dass sie nur von den gewünschten Teilnehmern verwendet werden können (s. \ref{sec:definition-sicherheitsrisiken}).

    \item \textbf{Datenschutz / Anonymität:}
    Der Zugang zu Daten, die auf privaten Chains festgehalten wurden, kann frei vom Betreiber kontrolliert werden. 
    Alle Daten, die auf einer öffentlichen Chain geschrieben wurden, sind für jeden einsehbar, so dass sensible Informationen off-chain gespeichert und übertragen  oder verschlüsselt werden sollten. 
    Es bestehen Designpattern, die dies erleichtern, sowie Layer 2 Lösungen, die Daten abgetrennt und von Layer 1 fernhalten können (s. \ref{sec:definition-layer-2-solutions}).

    Ebenso sind alle Transaktionen auf einer öffentlichen Chain öffentlich einsehbar, sodass die Anonymität der Teilnehmer nicht gewährleistet werden kann (s. \ref{sec:definition-account-based-vs-utxo-based-chain}).
\end{itemize}

% \paragraph{Transaction Relayers}
\label{sec:definition-transaction-relayers}
Um es Account zu ermöglichen Transaktionen zu tätigen, ohne ETH zu besitzen, besteht die Möglichkeit, dass ein Dritter die Transaktion für den Account tätigen und so auch bezahlen.\footcites[Vgl. hierzu und zum Folgenden sowie weiterführend][]{w24}[]{w26}
Dieser Dritte wird Transaction Relayer genannt.
Dabei wird die Transaktion vom Benutzer zwar signiert, dann aber nicht direkt ans Netzwer gesendet und so ausgeführt, sondern an den Dritten gesendet, der die Transaktion dann ausführt.
Es muss auf der Chain ein Smart Contract existieren, der die Transaktionen entgegennimmt, die Signatur prüft und dann die Transaktion durchführt.\footcite[Vgl.][]{w25}
Allerding können nur speziell dafür angepasste Funktionen eines Smart Contracts auf diese Weise ausgeführt werden, da der eigentliche „Ausführende” nicht der Sender ist und deshalb anders geprüft werden muss (nicht über \textit{msg.sender} sondern über bspw. \textit{msgSender()}).\footcites[Vgl.][]{w27}[]{w24}

\subsubsection{Sicherheitsrisiken}
\label{sec:definition-sicherheitsrisiken}
\todo[color=REVIEW]{
    Sicherheitsrisiken wie Reentrancy / Oracle / etc. Attack? - i.V.m. Audit \break
    --> Reentrancy: https://solidity-by-example.org/hacks/re-entrancy/ \break
    --> Reentrancy vermutlich keint großes Problem, da nur Kunden Transaktionen ausführen können \break
    --> Contract nicht öffentlich angepriesen
    }

\todoin[color=TODO]{
Fazit / Empfehlung welche Chain verwendet werden soll \break
--> Öffentlich, da sonst nicht dezentral und so unsicher (Änderung der Daten möglich / also quasi gleiches System wie jetzt)\break

Grundsätzlich viele Vorteile bei privater Chain (evtl. dann im Kontext, wenn mehrere Banken dies benutzen)

}

\subsection{Transaktionskosten}
\todo[color=TODO]{Transaktionskosten schreiben}
\todoin[color=NOTES]{
2 Möglichkeiten: \break
- Transfer wird immer noch von einer Adresse durchgeführt
- Tranfer wird vom Inhaber durchgeführt (So kann nur der Kunde selbst transferieren --> schauen wie es aktuell umgesetzt ist) --> Transaction Relays von Nöten \break
\break
\break

}
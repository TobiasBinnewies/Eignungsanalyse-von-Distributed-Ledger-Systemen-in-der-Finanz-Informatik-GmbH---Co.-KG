\noindent

\subsubsection{Öffentliche vs. private Chain}
\label{sec:oeffentlich-vs-privates-netzwerk}

\todoin[color=NOTES]{
    Öffentliches vs. privates Netzwerk: \break
    PRivates: \break
    - Nur für bestimmte Teilnehmer (in dem Falle dann nur FI) \break
    - wirds wirklich nur von einer Authority verwendet können die Daten geändert werden \break
    - "eigene" Regeln (bspw. kein Gas) \break
    \break
    Öffentliches: \break
    - Anonymität schwierig zu gewährleisten \break
    - Transaktionskosten \break
    - Daten können nicht geändert werden \break
}

Bei der Verwendung einer Blockchain gibt es die Möglichkeit, einer öffentlichen Chain „beizutreten“ oder dafür eine private Chain zu betreiben.\footcite[Vgl.][]{w10}
Eine öffentliche Chain ist für alle Teilnehmer offen und kann von jedem verwendet werden.\footcite[Vgl. hierzu und zum Folgenden][]{w11}
Eine privatee Chain hingegen ist nur für ausgewählte Teilnehmer zugänglich und wird i.d.R. von einer oder mehreren Organisationen / Unternehmen betrieben.
Die Auswahl der Art der Chain kann an den Punkten Sicherheit / Unveränderlichkeit, Leistung, Kosten, Berechtigungen und Datenschutz / Anonymität erfolgen:
\begin{itemize}
    \item \textbf{Sicherheit / Unveränderlichkeit:} 
    \todo[color=REVIEW]{Folgendes eher in Definition oder späterer Analyse?}
    Die Sicherheit und Unveränderlichkeit einer Blockchain wird durch ihren Konsensalgorithmus bestimmt. 
    Eine öffentliche Chain wird durch die Interaktion von Tausenden unabhängigen Nodes gesichert, die von Einzelpersonen und Organisationen auf der ganzen Welt betrieben werden. 
    Private Chains haben typischerweise eine kleine Anzahl von Nodes, die von einer oder wenigen Organisationen kontrolliert werden. 
    Diese Nodes können streng kontrolliert werden, aber nur wenige müssen kompromittiert werden, um die Chain umzuschreiben oder betrügerische Transaktionen durchzuführen.

    \item \textbf{Leistung:} 
    Bei privaten Chains können Hochleistungsnodes mit besondererer Hardware sowie ein anderer Konsensalgorithmus verwendet werden, um einen höheren Tansaktionsdurchsatz auf der Basisschicht (Layer 1) erreichen.
    Bei einer öffentlichen Chain kann ein hoher Durchsatz mit Hilfe von Layer 2 Skalierungslösungen erreicht werden.

    \item  \textbf{Kosten:}
    Die Kosten für den Betrieb einer privaten Chain spiegeln sich hauptsächlich in der Arbeit wider, die Chain einzurichten und zu verwalten, und den Servern zu betreiben, auf denen sie läuft. 
    Während es keine Kosten gibt, um sich mit dem Ethereum Mainnet zu verbinden, muss die Gas Fee (s. \ref{sec:definition-gas-fee}) für jede Transaktion in Ether bezahlt werden.
    Abhilfe können Transaction Relayers (s. \ref{sec:definition-transaction-relayers}) sein, sodass ein Endkunde diese Gebühr nicht selbst tragen muss.

    Einige Analysen haben gezeigt, dass die Gesamtkosten für den Betrieb einer Anwendung auf einer öffentlichen Chain niedriger sein können als beim Betrieb einer privaten Chain.\footcite[Vgl.][14]{q6}
    \todo[color=REVIEW]{Die Quelle genauer erläutern?}

    \item \textbf{Berechtigungen:}
    Bei privaten Chains können nur autorisierte Teilnehmer Nodes einrichten und Transaktionen durchführen.
    Bei öffentlichen Chains kann jeder Node einrichten und Transaktionen durchführen.
    So kann ebenfalls jeder auf jeden Contract zugreifen, also dessen gespeicherte Daten auslesen und Funktionen aufrufen.
    Daher müssen erstellte Contracts so implementiert werden, dass sie nur von den gewünschten Teilnehmern verwendet werden können (s. \ref{sec:definition-sicherheitsrisiken}).

    \item \textbf{Datenschutz / Anonymität:}
    Der Zugang zu Daten, die auf privaten Chains festgehalten wurden, kann frei vom Betreiber kontrolliert werden. 
    Alle Daten, die auf einer öffentlichen Chain geschrieben wurden, sind für jeden einsehbar, so dass sensible Informationen off-chain gespeichert und übertragen  oder verschlüsselt werden sollten. 
    Es bestehen Designpattern, die dies erleichtern, sowie Layer 2 Lösungen, die Daten abgetrennt und von Layer 1 fernhalten können (s. \ref{sec:definition-layer-2-solutions}).

    Ebenso sind alle Transaktionen auf einer öffentlichen Chain öffentlich einsehbar, sodass die Anonymität der Teilnehmer nicht gewährleistet werden kann (s. \ref{sec:definition-account-based-vs-utxo-based-chain}).
\end{itemize}

\noindent
Da sich der Scope dieser Arbeit in der FI aufhält und so das Betreiben einer privaten Chain nicht mehr dezentral, sondern zentral - weil sie dann nur von der FI betrieben werden würde - wird im weiteren Verlauf nur auf die Verwendung einer öffentlichen Chain eingegangen (s. weiterführend \ref{sec:ausblick}).
Grundsätzlich hätte das Betreiben einer privaten Chain viele Vorteile, da so die Transaktionskosten gespart werden könnten und die Daten nicht öffentlich einsehbar wären.
Allerdings würde sich so das System kaum von dem heutigen unterscheiden.

\todo[color=REVIEW]{Evtl. entfernen? --> dann auch Referenz entfernen}
\subsubsection{Sicherheitsrisiken}
\label{sec:definition-sicherheitsrisiken}
\todo[color=REVIEW]{
    Sicherheitsrisiken wie Reentrancy / Oracle / etc. Attack? - i.V.m. Audit \break
    --> Reentrancy: https://solidity-by-example.org/hacks/re-entrancy/ \break
    --> Reentrancy vermutlich keint großes Problem, da nur Kunden Transaktionen ausführen können \break
    --> Contract nicht öffentlich angepriesen
    }

\todoin[color=TODO]{
Fazit / Empfehlung welche Chain verwendet werden soll \break
--> Öffentlich, da sonst nicht dezentral und so unsicher (Änderung der Daten möglich / also quasi gleiches System wie jetzt)\break

Grundsätzlich viele Vorteile bei privater Chain (evtl. dann im Kontext, wenn mehrere Banken dies benutzen)

}

\subsubsection{Transaktionskosten und -durchsatz}
\todo[color=REVIEW]{Anordnung überarbeiten}
\todoin[color=NOTES]{
2 Möglichkeiten: \break
- Transfer wird immer noch von einer Adresse durchgeführt
- Tranfer wird vom Inhaber durchgeführt (So kann nur der Kunde selbst transferieren --> schauen wie es aktuell umgesetzt ist) --> Transaction Relays von Nöten \break
\break
\break
\break
Polygon / Layer 2 \break

}

\paragraph{Layer 2 Lösungen}
\label{sec:definition-layer-2-solutions}

Layer 2 ist ein grundsätzlicher Begriff für Lösungen, die „off-chain” - also außerhalb der eigentlichen Blockchain (Layer 1) - betrieben werden, dort die Transaktionen gebündelt und nur die Ergebnisse auf die Layer 1 Chain geschrieben werden.\footcite[Vgl.][]{w18}
Dabei gibt es verschiedene Ansätze, unter anderem die Folgenden:
\begin{itemize}
    \item \textbf{Sidechains:} 
    Eine Sidechain ist eine eigene Blockchain, die mit der Layer 1 Chain verbunden ist und so parallel zu dieser läuft. 
    Diese Chain hat ihre eigenen Regeln zum Thema Konsensalgorithmus und Blöcken.\footcite[Vgl. hierzu und weiterführend][]{w20}

    \noindent
    Da hier eine eigene Blockchain verwendet wird, ist diese Lösung sehr flexibel und kann für viele verschiedene Anwendungsfälle verwendet werden.
    Allerdings wird nicht die Sicherheit der Layer 1 Chain übernommen, sondern es muss der Sidechain vertraut werden, da ein eigener Konsensalgorithmus verwendet wird.\footcite[Vgl.][]{w21}


    \item \textbf{State Channels:}
    State Channels ermöglicht es, vielfache Transaktionen „off-chain” zu tätigen und dabei nur zwei Transaktionen auf die Layer 1 Chain zu schreiben - das Öffnen und Schließen des Channels.
    Es gibt zwei Arten von Channels: Payment Channels und State Channels.\footcite[Vgl. hierzu und zum Folgenden sowie weiterführend][]{w19}

    \noindent
    Payment Channels sind sicher, da sie den Konsensalgorithmus der Layer 1 Chain verwenden, können allerdings nur für spezifische Anwendungszwecke verwendet werden (bspw. Coin- oder Token-Transaktionen).

    \noindent
    State Channels können für beliebige Transaktionen verwendet werden, sind allerdings unsicherer, da die Validität der Transaktionen nur durch die Teilnehmer des Channels geprüft werden und nicht durch die Layer 1 Chain.


    \item \textbf{Rollups:}
    Rollups führen die Transaktionen ebenfalls „off-chain” aus.
    Die Ergebnisse werden dann in einem Layer 1 Block gesammelt und so in die Layer 1 Chain geschrieben.\footcite[Vgl. hierzu und zum Folgenden][]{w18} 
    So werden Rollups mit der Layer 1 Chain gesichert.
    Es gibt zwei Arten von Rollups: Optimistic Rollups und Zero Knowledge Rollups.

    \noindent
    Optimistic Rollups gehen davon aus, dass Off-Chain-Transaktionen gültig sind und veröffentlichen keine Nachweise für die Gültigkeit der auf der Chain veröffentlichten Transaktionsbatches.\footcite[Vgl. hierzu und zum Folgenden sowie weiterführend][]{w22}
    Nachdem ein Rollup-Batch auf Ethereum eingereicht wurde, gibt es ein Zeitfenster (die sogenannte Challenge-Periode), in dem jeder die Ergebnisse einer Rollup-Transaktion durch Berechnung eines Betrugsnachweises anfechten kann („Fraud Proof”).
    Wenn der Betrugsnachweis erfolgreich ist, wird die Transaktion neu ausgeführt und der Zustand des Rollups entsprechend aktualisiert.

    \noindent
    In ZK-Rollups wird ein Stapel (engl. „Batch”) von Transaktionen auf dem Ethereum-Netzwerk auf Korrektheit überprüft.\footcite[Vgl. hierzu und zum Folgenden sowie weiterführend][]{w23}
    Nach dieser Prüfung wird der Stapel von Transaktionen als endgültig betrachtet, genau wie jede andere Ethereum-Transaktion. Dafür wird ein kryptographischer Gültigkeitsnachweise (allgemein als Zero-Knowledge-Nachweise bezeichnet) verwendet. 
    Bei jedem Stapel von Off-Chain-Transaktionen generiert der ZK-Rollup-Betreiber einen Gültigkeitsnachweis für diesen Stapel. Sobald der Nachweis generiert ist, wird er an Ethereum übermittelt, um den Roll-up-Stapel in der Layer 1 Chain einzubinden.
\end{itemize}

\todo[color=TODO]{Fazit zu Layer 2 Lösungen schreiben}
\bigbreak
\noindent
Um mit dem hohen Datenvolumen der Sparkassen fertigzuwerden, müssen Layer 2 Lösungen verwendet werden. 

\paragraph{Transaction Relayers}
\label{sec:definition-transaction-relayers}
Um es Account zu ermöglichen Transaktionen zu tätigen, ohne ETH zu besitzen, besteht die Möglichkeit, dass ein Dritter die Transaktion für den Account tätigen und so auch bezahlen.\footcites[Vgl. hierzu und zum Folgenden sowie weiterführend][]{w24}[]{w26}
Dieser Dritte wird Transaction Relayer genannt.
Dabei wird die Transaktion vom Benutzer zwar signiert, dann aber nicht direkt ans Netzwer gesendet und so ausgeführt, sondern an den Dritten gesendet, der die Transaktion dann ausführt.
Es muss auf der Chain ein Smart Contract existieren, der die Transaktionen entgegennimmt, die Signatur prüft und dann die Transaktion durchführt.\footcite[Vgl.][]{w25}
Allerding können nur speziell dafür angepasste Funktionen eines Smart Contracts auf diese Weise ausgeführt werden, da der eigentliche „Ausführende” nicht der Sender ist und deshalb anders geprüft werden muss (nicht über \textit{msg.sender} sondern über bspw. \textit{msgSender()}).\footcites[Vgl.][]{w27}[]{w24}

\todo[color=TODO]{Fazit zu Transaction Relayers schreiben}
\bigbreak
\noindent
Transaction Relayers wären sinnvoll, wenn ein Kunde selbst über das Netzwerk bpsw. Überweisungen o. ä. durchführen kann.
Da allerdings nur die FI selbst Transaktionen durchführt - ein Endkunde würde dann durch die Services der Sparkassen und so der FI das System verwenden - und diese selbst die Kosten für die Transaktionen tragen muss, ist es nicht notwendig Transaction Relayers zu verwenden.
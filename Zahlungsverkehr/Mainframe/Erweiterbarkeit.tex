\subsubsection{Erweiterbarkeit}
\label{sec:Erweiterbarkeit}

Sollten Finanzprodukte in Smart Contracts umgesetzt werden oder sich die Funktionsweise der Optimistic Rollups in Zukunft ändern, ist es wichtig, dass diese Funktionen erweiterbar / weiterentwickelbar sind.
Die Grundidee einer Blockchain basiert allerdings darauf, dass einmal geschriebene Daten - und so auch Smart Contract - nicht mehr verändert werden können (sie sind „immutable”).\footcite[Vgl.][]{w30}
Es gibt drei Möglichkeiten, wie Smart Contracts erweitert werden können:\footcite[Vgl. hierzu und zum Folgenden sowie weiterführend][]{w29}
\begin{itemize}
    \item \textbf{Contract Migration:}
    Bei der Contract Migration löst ein neuer Contract den alten ab.
    Alle Daten, die im alten Contract gespeichert wurden, müssen also in den neuen Contract übertragen werden.
    Die Datamigration ist meist sehr aufwendig und mit hohen Gas-Kosten verbunden.

    \item \textbf{Data Seperation:}
    Bei der Data Seperation werden Daten und Logik in zwei verschiedene Contracts aufgeteilt.
    So kann der „Logik-Contract” ersetzt werden, ohne dass die Daten migriert werden müssen.
    Hier müssen aufwendige Sicherheitsmaßnahmen getroffen werden, um zu gewährleisten, dass nur der Logik-Contract die Daten verändern kann.

    \item \textbf{Proxy:}
    Bei der Verwendung eines Proxys wird ein zusätzlicher Contract verwendet, der als Schnittstelle zu dem eigentlichen Contract dient.
    Es ist möglich, dass ein Contract die Funktionen eines anderen Contracts in seinem Kontext - also mit seinem Speicher - aufruft.
    Der Proxy Contract hält also den Speicher, während der dahintergeschaltete Contract die Logik enthält.
    So kann der Proxy Contract einfach ausgetauscht werden, ohne dass die Daten migriert werden müssen.

    \noindent
    Bei Proxys gibt es mehrere Architekturen, die sich in der Art und Weise, wie der Proxy mit dem dahintergeschalteten Contract interagiert, unterscheiden.
    Häufig kommt zusätzlich ein Admin Contract zum Einsatz, der die Berechtigung hat, den Logik-Contract zu ändern.
\end{itemize}

\noindent
Da mehrere Finanzprodukte in Smart Contract abgebildet werden könnten, bietet es sich an für jedes Produkt einen eigenen Contract zu verwenden und diese dann über einen gemeinsamen Proxy anzusprechen.

\todoin[color=NOTES]{
    - Optimitic Rollups als Ablöse für Großrechner: \break
    - Einsatz auf bspw. Cloud Systemen \break
        --> Vorteile es bleibt klar, auf welchen Systemen die Daten liegen \break
        --> Es kann so verhältnismäßig einfach auf andere Anbieter gewechselt werden \break

    Es gäbe dennoch die Möglichkeit - je nach konkreter Implementierung - bestimmte Sicherheitsmechanismen einzubauen, um bspw. gegen Geldwäsche oder fehlerhaft Buchungen vorzugehen. \todo[color=TODO]{Ausführlicher \& Besipiele / konkrete Idee der Implementierung}
    Außerdem ist ein AuditLog automatisch vorhanden, da alle Transaktionen in der Blockchain gespeichert werden.
    Aufbauend darauf können Smart Contracts verwendet werden, um bestimmte Finanzprodukte (z.B. Sparverträge, Kredite, ...) oder auch Multisign (per Multisign Contracts) abzubilden und so zu automatisieren. \todo[color=TODO]{Ausführlicher \& Besipiele / konkrete Idee der Implementierung}
}

\subsection{DLS als Mainframersatz}
Um den Mainframe wirklich ablösen zu können, reicht es nicht aus lediglich die Datenbank zu ersetzen, es muss auch die Verarbeitung und Prüfung der Daten übernommen werden.
Kurzum müssen also die COBOL- und Java-Jobs sowie die DOINGs, die akutell auf dem Mainframe laufen, ersetzt werden.

\noindent
Eine Möglichkeit wäre es all diese Jobs durch Smart Contracts zu ersetzen.
Allerdings wäre so alle Verarbeitungen und Prüfungen öffentlich einsehbar, was sensible Daten (wie bspw. Name oder IBAN) offenlegen würde.
Außerdem wäre dies durch die Transaktionskosten sehr teuer.

\noindent
Da durch das hohe Datenaufkommen der Sparkassen die Verarbeitung der Daten sehr performant sein muss, ist es nicht ebenfalls möglich, die Verarbeitung der Daten auf der Blockchain selbst durchzuführen.
Wie bereits beschrieben muss also eine Layer 2 Lösung verwendet werden (s. \ref{sec:definition-layer-2-solutions}).
Da die FI die einzige Instanz ist, die Transaktionen durchführt und so keine gesonderte Prüfung der Transaktionen notwendig ist, bietet sich hier ein (modifiziertes) Optimistic Rollup an. 
(Die Challange-Periode ist nicht von Nöten, es wird also immer der durch das Rollups kreierte Batch verwendet.)
Die FI wickelt also den Zahlungsverkehr weiterhin auf eigenen Systemen ab und übersendet nur die Ergebnisse an die Layer 1 Chain.
So werden die Prüfungen und Verarbeitungen der Daten auch nicht öffentlcih ausgeführt und sind so nicht einsehbar.

\noindent
Bestimmte Finanzprodukte lassen sich dennoch gut durch Smart Contracts abbilden und so automatisieren.
Darunter zählen z. B. automatische Zinsvergabe am Monats- oder Jahresabschluss oder auch Termin- oder Daueraufträge.

\subsubsection{Optimistic Rollups in Verbindung mit Cloud Computing}
Eine mögliche Alternative den Mainframe abzulösen, ist die Verwendung von Cloud Computing.
Cloud Computing definiert sich nach dem National Institute of Standards and Technology (NIST) durch die folgenden fünf Eigenschaften:\footcite[Vgl. hierzu und zum Folgenden][5]{q11}
\begin{itemize}
    \item \textbf{On-Demand Self-Service:} 
    Der Nutzer kann die benötigten Ressourcen (bspw. Rechenleistung, Speicherplatz, ...) selbstständig und ohne meschliche Interaktion mit dem Anbieter bereitstellen.
    \item \textbf{Broad Network Access:}
    Die Ressourcen sind über das Netzwerk verfügbar und können von verschiedenen Plattformen (bspw. Desktop, Laptop, Smartphone, ...) abgerufen werden.
    \item \textbf{Resource Pooling:}
    Die Ressourcen des Anbieters werden gebündelt und können so von mehreren Nutzern verwendet werden.
    \item \textbf{Rapid Elasticity:}
    Die Ressourcen können schnell und dynamisch an die Bedürfnisse des Nutzers angepasst werden.
    \item \textbf{Measured Service:}
    Die Nutzung der Ressourcen wird quntitativ und qualitativ überwacht, so dass eine nutzungsabhängige Abrechnung sowie eine Validierung der Dienstqulität möglich ist.
\end{itemize}

\noindent
Der größte Vorteil für die FI wäre hier die gute Skalierbarkeit der Ressourcen und die nutzungsabhängige Abrechnung.
\bigbreak
\noindent
Ein größer Nachteil dieser Technologie ist die Datensicherheit, da sich diese in der Hand des Anbieters befinden und so nicht kontrolliert werden kann.\footcite[Vgl. hierzu und zum Folgenden][99]{q10}
Es besteht generell keine Kontrolle über die Art und Weise der Speicherung.
So ist ein Wechsel des Anbieters sehr aufwendig, da die Daten möglicherweise nicht einfach migriert werden können.\footcite[Vgl.][]{w28}

\noindent
Hier können DLS Abhilfe schaffen.
So können die Optimistic Rollups auf einem Cloud System ausgeführt werden, die Daten aber dennoch in der Blockchain gespeichert werden.
So sind die tranzparent und nach eigenen Vorstellungen gespeichert, aber die Verarbeitung der Daten kann auf einem Cloud System erfolgen.
Darüber hinaus können die Daten nicht nur vom verwendeten Anbieter selbst aus eingesehen werden und eine Migration ist so einfacher möglich.
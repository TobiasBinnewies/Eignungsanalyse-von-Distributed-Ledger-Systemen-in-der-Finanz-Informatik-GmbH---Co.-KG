\subsection{DLS als Mainframersatz}
Um den Mainframe wirklich ablösen zu können, reicht es nicht aus lediglich die Datenbank zu ersetzen.
Es muss auch die Verarbeitung und Prüfung der Daten übernommen werden.
Kurzum müssen also die COBOL- und Java-Jobs sowie die DOINGs, die akutell auf dem Mainframe laufen, ersetzt werden.

\noindent
Eine Möglichkeit wäre es, all diese Jobs durch Smart Contracts zu ersetzen.
Allerdings wären so alle Verarbeitungen und Prüfungen öffentlich einsehbar, was sensible Daten (wie bspw. Name oder IBAN) offenlegen würde.
Außerdem wäre dies durch die Transaktionskosten sehr teuer.

\noindent
Da durch das hohe Datenaufkommen der Sparkassen die Verarbeitung der Daten sehr performant sein muss, ist es ebenfalls nicht möglich, die Verarbeitung der Daten auf der Blockchain selbst durchzuführen.
Wie bereits beschrieben muss also eine Layer 2 Lösung verwendet werden (s. \ref{sec:definition-layer-2-solutions}).
Da die FI die einzige Instanz ist, die Transaktionen durchführt und so keine gesonderte Prüfung der Transaktionen notwendig ist, bietet sich hier ein (modifiziertes) Optimistic Rollup an. 
(Die Challange-Periode ist nicht erforderlich, es wird also immer der durch das Rollups kreierte Batch verwendet.)
Die FI wickelt also den Zahlungsverkehr weiterhin auf eigenen Systemen ab und übersendet nur die Ergebnisse an die Layer 1 Chain.
So werden die Prüfungen und Verarbeitungen der Daten auch nicht öffentlich ausgeführt und sind nicht einsehbar.

\noindent
Bestimmte Finanzprodukte lassen sich dennoch gut durch Smart Contracts abbilden und so automatisieren.
Darunter zählen z. B. automatische Zinsvergaben am Monats- oder Jahresabschluss oder auch Termin- oder Daueraufträge.

\subsubsection{Optimistic Rollups in Verbindung mit Cloud Computing}
Eine mögliche Alternative den Mainframe abzulösen, ist die Verwendung von Cloud Computing.
Cloud Computing wird nach dem National Institute of Standards and Technology durch die folgenden fünf Eigenschaften definiert:\footcite[Vgl. hierzu und zum Folgenden][5]{q11}
\begin{itemize}
    \item \textbf{On-Demand Self-Service:} 
    Der Nutzer kann die benötigten Ressourcen (bspw. Rechenleistung, Speicherplatz, ...) selbstständig und ohne menschliche Interaktion mit dem Anbieter bereitstellen.
    \item \textbf{Broad Network Access:}
    Die Ressourcen sind über das Netzwerk verfügbar und können von verschiedenen Plattformen (bspw. Desktop, Laptop, Smartphone, ...) abgerufen werden.
    \item \textbf{Resource Pooling:}
    Die Ressourcen des Anbieters werden gebündelt und können so von mehreren Nutzern verwendet werden.
    \item \textbf{Rapid Elasticity:}
    Die Ressourcen können schnell und dynamisch an die Bedürfnisse des Nutzers angepasst werden.
    \item \textbf{Measured Service:}
    Die Nutzung der Ressourcen wird quantitativ und qualitativ überwacht, so dass eine nutzungsabhängige Abrechnung sowie eine Validierung der Dienstqualität möglich ist.
\end{itemize}

\noindent
Der größte Vorteil für die FI wäre hier die gute Skalierbarkeit der Ressourcen und die nutzungsabhängige Abrechnung.
\bigbreak
\noindent
Ein größer Nachteil dieser Technologie ist die Datensicherheit, da sich diese in der Hand des Anbieters befindet und so nicht kontrolliert werden kann.\footcite[Vgl. hierzu und zum Folgenden][99]{q10}
Es besteht generell keine Kontrolle über die Art und Weise der Speicherung.
So ist ein Wechsel des Anbieters sehr aufwendig, da die Daten möglicherweise nicht einfach migriert werden können.\footcite[Vgl.][]{w28}

\noindent
Hier können DLS Abhilfe schaffen.
So können die Optimistic Rollups auf einem Cloud System ausgeführt, die Daten aber dennoch in der Blockchain gespeichert werden.
So sind sie tranzparent und nach eigenen Vorstellungen gespeichert, aber die Verarbeitung der Daten kann auf einem Cloud System erfolgen.
Darüber hinaus können die Daten nicht nur vom verwendeten Anbieter selbst aus eingesehen werden - da sie sich im öffentlcihen Netzwerk befinden - und eine Migration ist so einfacher möglich.
Man ist also nicht stark an einen Anbieter gebunden.

\subsubsection{Erweiterbarkeit}
\label{sec:Erweiterbarkeit}

Sollten Finanzprodukte in Smart Contracts umgesetzt werden oder sich die Funktionsweise der Optimistic Rollups in Zukunft ändern, ist es wichtig, dass diese Funktionen erweiterbar / weiterentwickelbar sind.
Die Grundidee einer Blockchain basiert allerdings darauf, dass einmal geschriebene Daten - und so auch Smart Contract - nicht mehr verändert werden können (sie sind „immutable”).\footcite[Vgl.][]{w30}
Es gibt drei Möglichkeiten, wie Smart Contracts erweitert werden können:\footcite[Vgl. hierzu und zum Folgenden sowie weiterführend][]{w29}
\begin{itemize}
    \item \textbf{Contract Migration:}
    Bei der Contract Migration löst ein neuer Contract den alten ab.
    Alle Daten, die im alten Contract gespeichert wurden, müssen also in den neuen Contract übertragen werden.
    Die Datamigration ist meist sehr aufwendig und mit hohen Gas-Kosten verbunden.

    \item \textbf{Data Seperation:}
    Bei der Data Seperation werden Daten und Logik in zwei verschiedene Contracts aufgeteilt.
    So kann der „Logik-Contract” ersetzt werden, ohne dass die Daten migriert werden müssen.
    Hier müssen aufwendige Sicherheitsmaßnahmen getroffen werden, um zu gewährleisten, dass nur der Logik-Contract die Daten verändern kann.

    \item \textbf{Proxy:}
    Bei der Verwendung eines Proxys wird ein zusätzlicher Contract verwendet, der als Schnittstelle zu dem eigentlichen Contract dient.
    Es ist möglich, dass ein Contract die Funktionen eines anderen Contracts in seinem Kontext - also mit seinem Speicher - aufruft.
    Der Proxy Contract hält also den Speicher, während der dahintergeschaltete Contract die Logik enthält.
    So kann der Proxy Contract einfach ausgetauscht werden, ohne dass die Daten migriert werden müssen.

    \noindent
    Bei Proxys gibt es mehrere Architekturen, die sich in der Art und Weise, wie der Proxy mit dem dahintergeschalteten Contract interagiert, unterscheiden.
    Häufig kommt zusätzlich ein Admin Contract zum Einsatz, der die Berechtigung hat, den Logik-Contract zu ändern.
\end{itemize}

\noindent
Da mehrere Finanzprodukte in Smart Contract abgebildet werden könnten, bietet es sich an für jedes Produkt einen eigenen Contract zu verwenden und diese dann über einen gemeinsamen Proxy anzusprechen.